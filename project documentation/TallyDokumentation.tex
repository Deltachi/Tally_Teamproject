%------------------------------------------------------------------------------------------------
%
% LaTeX document Tally
%
%------------------------------------------------------------------------------------------------

\documentclass[11pt,a4paper]{article} % weitere Option: twoside

%------------------------------------------------------------------------------------------------

% weitere Pakete laden
        % Kodierung einstellen
\usepackage[ngerman]{babel}
\usepackage[T1]{fontenc}
\usepackage{graphicx}
\usepackage{ngerman}
\usepackage{listings}
\usepackage{color}
\usepackage[utf8x]{inputenc}


\definecolor{dkgreen}{rgb}{0,0.6,0}
\definecolor{gray}{rgb}{0.5,0.5,0.5}
\definecolor{mauve}{rgb}{0.58,0,0.82}

\lstset{numbers=left,
	numberstyle=\tiny,
	numbersep=5pt,
	breaklines=true,
	showstringspaces=false,
	frame=l ,
	xleftmargin=15pt,
	xrightmargin=15pt,
	basicstyle=\ttfamily\scriptsize,
	stepnumber=1,
	keywordstyle=\color{blue},          % keyword style
  	commentstyle=\color{dkgreen},       % comment style
  	stringstyle=\color{mauve}         % string literal style
}






% weitere Formatierung
\frenchspacing                         % Kein zus�tzlichen Abstand nach Punkt (.)
\setlength{\parindent}{0cm}            % keine Einr�ckung bei Beginn eines Absatzes
\setlength{\parskip}{1.5ex plus 0.5ex minus 0.5ex}  % mehr Absatzabstand


%------------------------------------------------------------------------------------------------
%
%  AB HIER SOLLEN �NDERUNGEN VORGENOMMEN WERDEN
%
%------------------------------------------------------------------------------------------------

% Hier sucht man sich den gew�nschten Stil f�r Kopf- bzw. Fu�zeile aus.
%\pagestyle{empty}                     % keine Kopf- und Fu�zeilen
\pagestyle{plain}                      % nur Seitenzahlen
%\pagestyle{headings}                  % Aktivieren f�r Kopf- und Fu�zeilen



\title{\normalfont\bfseries{Tally\\ digitale Strichliste am Raspberry Pi}\\---}
\author{Nicolai Tegtmeier \\ Dominik Scheffler \\ Philipp Frieling \\ Sebastian Reinke}
\date{23.06.2015}


%------------------------------------------------------------------------------------------------

\begin{document}

%------------------------------------------------------------------------------------------------

%\maketitle  % Titel erzeugen, verwendet die Angaben aus \title, \author und \date

%\tableofcontents % Um Inhaltsverzeichnis zu erzeugen

\begin{titlepage}
	\maketitle
	\includegraphics[width=12cm]{TallyLogo.png}
\end{titlepage}
%------------------------------------------------------------------------------------------------

\tableofcontents
\newpage

\section{Projektvorfeld}
\label{Grundlagen}


\subsection{Kaffeestrichliste bisher}

In kleineren Unternehmen und Arbeitsgruppen gibt es oft eine zentralen Kaffee/ Getr\"anke -automaten und Snacks, an der sich jeder Mitarbeiter bedienen kann. Um die Getr\"anke und Snacks kaufen zu k\"onnen wird eine Strichliste auf Papier, meist in der N\"ahe des Automaten oder der Snacks, angebracht damit sich jeder Mitarbeiter eintragen kann, was er gekauft hat. Diese Liste wird oft zur besseren Verwaltung in eine Datenbank oder Tabelle eingepflegt. Hier passiert es jedoch oft das die Strichliste nicht sehr leserlich ist, und es bei der \"Ubertragung zu fehlern kommen kann, so dass ein Minus in der Kasse entsteht. Au/ss{}erdem ist diese Methode f\"ur den Verantwortlichen sehr zeitaufwendig.
\par
Um diesem Problem entgegen zu wirken, erarbeiten wird eine neue Methode um diese Daten 'digital' speichern zu k\"onnen.

%------------------------------------------------------------------------------------------------

\section{Rahmenbedingungen}


\subsection{Aufgabe - die neue Kaffeestrichliste}
\label{SchriftAnpassen}

Um eine m\"oglichst Energiesparende und handliche L\"osung zu finden, sollte die neue Strichliste auf einem Raspberry Pi 2 realisiert werden. Zun\"achst soll eine passende Benutzeroberfl\"ache f\"ur den Raspberry entwickelt werden, damit der Benutzer am Raspberry selber seine Eink\"aufe verbuchen kann. Dies soll mithilfe der integrierten Entwicklungsumgebung 'Qt Creator', die besonders zur Entwicklung von von plattformunabh\"angigen C++ Programmen gedacht ist, entwickelt werden.
\par
Mithilfe eines Webservers, der ebenfalls auf dem Raspberry l\"auft, soll die Administration von jedem Computer im selben Netzwerk \"uber eine Weboberfl\"ache m\"oglich sein. Hier sollen auch die Benutzer ihre Daten einsehen und \"andern k\"onnen. Dazu wird ein MySQL/ SQLite Datenbank, sowie PHP Unterst\"utzung ben\"otigt. Mittels eines Barcodescanners soll es m\"oglich sein am Raspberry Produkte ein zu scannen.


\section{Ziele des Projekts}
Die neue 'digitale Kaffeestrichliste' wird auf Basis eines Raspberry Pi 2 entwickelt, um die alte Strichliste abzul\"osen. Dazu wurden folgende zu erreichende Ziele festgelegt:
\subsection{Zielbestimmung}
\label{Ausrichtung}

\subsubsection{Der Benutzer Account}
\"Uber den Benutzer Account soll der Benutzer sich am Raspberry selbst und an der Weboberfl\"ache anmelden. Am Raspberry selbst k\"onnen Getr\"anke und Snacks gekauft werden. Mithilfe der Weboberfl\"ache kann der Benutzer Statistiken einsehen und sein Passwort \"andern.

\subsubsection{Das Programm}
Das Programm das auf dem Raspberry l\"auft aktualisiert die Anzahl der Produkte im Lagerbestand automatisch, gibt Meldungen bei zu geringem Warebestand aus und gibt generelle Fehlermeldungen aus

\subsubsection{Der Administrator Account}
Die Administration mithilfe des Administrator Accounts findet ausschliesslich \"uber die Weboberfl\"ache statt. Hier kann der Administrator Accounts hinzuf\"ugen und verwalten, Waren hinzuf\"ugen und verwalten und den Lagerbestand verwalten.

\subsection{Produkteinsatz}
\subsubsection{Anwendungsbereiche}
Mitarbeiter, beziehungsweise die Administratoren, k\"onnen eine Strichliste 'digital' anlegen. Diese fasst den zu bezahlenden Betrag f\"ur die Mitarbeiter zusammen.

\subsubsection{Zielgruppen}
Personengruppen und Unternehmen bei denen Getr\"anke und Snacks privat f\"ur alle angeboten und privat bezahlt werden, jedoch Schwierigkeiten mit der Handhabung einer herk\"ommlichen Strichliste haben und den Bestand an Getr\"anken und Snacks erfassen wollen.

\subsubsection{Betriebsbedingungen}
Die Strichliste soll m\"o
glichst Wartungsarm und einen niedrigen Stromverbrauch haben. Desweiteren soll sie t\"aglich vierundzwanzig Stunden laufen.

\subsection{Produktumgebung}
Das Produkt kann unabh\"angig an jedem Ort betrieben werden, solange eine Stromquelle in der n\"ahe ist.

\subsubsection{Software}
Die einzige Software die vom Benutzer ben\"otigt wird ist ein aktueller Webbrowser.

\subsubsection{Hardware}
Der Benutzer kann mit jedem Internetf\"ahigen Ger\"at die Weboberfl\"ache erreichen.

\subsubsection{Orgware}
Der Administrator kann die Betriebsparameter konfigurieren.

\subsection{Produktfunktion}
\subsubsection{Benutzerfunktion}
Ein im System registrierter Benutzer kann das System erst nutzen, wenn er angemeldet ist.
Nur ein Administrator kann einen Benutzer anlegen und ihm einen Benutzernamen sowie Passwort zuwesien. Sobald ein Benutzer registriert ist kann er sich sowohl am Raspberry als auch an der Weboberfl\"ache anmelden. Daf\"ur ben\"otigt er seinen Benutzernamen und sein Passwort. Abmelden ist bei beiden Oberfl\"achen jederzeit m\"oglich.
\par
Der Administrator kann \"uber die Weboberfl\"ache Produkte hinzuf\"ugen,entfernen und deren Details \"andern. Benutzer k\"onnen \"uber die Weboberfl\"ache Favoriten einrichten und Statistiken einsehen.
\par
Am Raspberry kann der registrierte und angemeldete Benutzer ein Produkt aus einer Liste w\"ahlen oder mithilfe eines Barcodescanners ein Prdoukt einscannen, welches im Warenkorb erscheint. Desweiteren kann er die Anzahl der Produkte erh\"ohen und seinen gesamten Warenkorb einsehen. Wenn er alle Waren im Warenkorb hat, kann er zur Kasse und die Waren durch einen Klick bezahlen, wodurch sein Konto belastet wird. Ein Abbruch oder das erreichen des Hauptmen\"us ist jederzeit m\"oglich.

\subsection{Programmfunktion}
Beim Kauf eines Produktes aktualisiert das Programm automatisch den Warenbestand. Bei geringem Warenbestand wird eine Warnmeldung ausgegeben und bei fehlen des Artikels wird dieser entfernt.

\subsection{Serverfunktion}
Auf dem Raspberry soll ein Apache 2 Server mit PHP Unterst\"utzung laufen.
\subsubsection{Datenbank}
Als Datenbank soll die resourcenschonendere SQLite Datenbank verwendet werden

\subsection{Benutzerschnittstelle}
Die Bedienung des Raspberry erfolgt \"uber einen eingebauten Touchscreen am Raspberry. Au\ss()erdem wird ein Barcodescanner installiert um Produkte einscannen zu k\"onnen.

\begin{center}
\begin{large}
\textbf{Zentrierter Text hingegen ist schon eher nützlich.}
\end{large}
\end{center}


\section{Meilensteine des Projektes}
\label{Umbrueche}

\subsection{Raspberry Pi 2 - Die Hardware mit der passenden Software}
Mit dem Raspberry Pi 2 war schon im Projektvorfeld eine passende Basis f\"ur das Projekt gelegt. Ausser dem Raspberry Pi 2 war zun\"achst ein 3,5 Zoll Touchscreen der Firma admatec mit dem Namen C-Berry vorhanden, das mittels eines Adapterboards an der GPIO Stiftleiste des Raspberry angesteckt wird. Das Display besitzt eine Aufl\"osung von 320x240 Pixeln und wird komplett vom Raspberry mit Strom versorgt.
\par
\subsubsection{Betriebssystem und Verbindung zu anderen Rechnern}
Als erster Schritt wurde in Bezug auf die Hardware und die Software die auf dem Raspberry laufen soll entschieden das es von Vorteil ist das speziell f\"ur den Raspberry zugeschnittene Raspbian Betriebssystem zu verwenden. Als Alternative stand eine Ubuntu Version ,speziell f\"ur ARM-Prozessoren entwickelt, zu Verf\"ugung. Hier stellte sich aber schnell nach einigen Recherchen heraus das Raspbian durch die Optimierung auf den Raspberry und den guten Support am besten f\"ur das Projekt geeignet ist. Um die weitere Einrichtung des Raspberrys zu vereinfachen wurde, neben der vorkonfigurierten und installierten SSH Verbindung zur Verwaltung \"uber ein Konsolenprogramm (Windows: Putty), eine freie Implementierung des Remote Desktop Protocols f\"ur Linux Names 'xrdp' installiert.
\begin{frame}

\begin{lstlisting}
sudo apt-get install xrdp
\end{lstlisting}

\end{frame}
Damit ist eine einfache Remote Desktopverbindung zum Raspberry m\"oglich.
\par
\subsubsection{Der Webserver}
Als n\"achster Schritt wurde ein vollst\"andiger Webserver auf dem Raspberry eingerichtet. Hierzu wurde zun\"achst ein Apache Server der Version 2 installiert
\begin{frame}

\begin{lstlisting}
sudo apt-get install apache2
\end{lstlisting}

\end{frame}
 der \"uber eine hohe Stabilit\"at und Geschwindigkeit verf\"ugt und serverseitig die Skriptsprache PHP unterst\"utzt. Im Anschluss wurde das PHP5 Paket installiert um eine volle Unterst\"utzung f\"ur die kommende Website zu gew\"ahrleisten.
\par
\subsubsection{Die passende Datenbank}
Nun stand die Auswahl des passenden Datenbankverwaltungssystems an. Zur Auswahl standen die Systeme MySQL und SQLite. Bei Recherchen zu den beiden Datenbanken stellte sich schnell heraus das die MySQL Datenbank zwar auf dem Raspberry lauff\"ahig ist aber keine hohe Stabilit\"at und sehr resourcenlastig auf dem Raspberry l\"auft. Daher wurde das SQLite Datenbanksystem ausgew\"ahlt.
\begin{frame}

\begin{lstlisting}
apt-get install sqlite3
apt-get install php5-sqlite
\end{lstlisting}

\end{frame}
 SQLite ist eine relationale Datenbank und ben\"otigt im Gegensatz zu MySQL keine st\"andig laufende Software da die Datenbank aus einer einzigen, wenigen Kilobyte grossen, Datei besteht.
\par
\subsubsection{Der Touchscreen}
Als erstes wurde der Treiber für den bereitgestellte C-Berry Touchscreen installiert
\begin{frame}

\begin{lstlisting}
wget http://www.airspayce.com/mikem/bcm2835/bcm2835-1.36.tar.gz
tar zxvf bcm2835-1.36.tar.gz
cd bcm2835-1.36
./configure
make
sudo make check
sudo make install
\end{lstlisting}

\end{frame}
und im anschluss die Beispielsoftware zum anzeigen von Bildern installiert
\begin{frame}

\begin{lstlisting}
wget http://admatec.de/sites/default/files/downloads/C-Berry.tar.gz
tar zxvf C-Berry.tar.gz
cd C-Berry/SW/tft_test
make
sudo ./tft_test
\end{lstlisting}

\end{frame}

Nach ersten Tests mit dem C-Berry Touchscreen fiel auf das es weniger gut als Prim\"ares Display geeignet ist da vom Desktop ein Screenshot erstellt wird und dieser auf dem Display  ausgegeben wird. Starke Verz\"ogerungen im Bildaufbau und eine schlechte Bedienung mithilfe des Touchscreens waren die Folge.
\par
Um dem Problem entegegen zu wirken musste ein neuer Touchscreen mit besserer Anbindung an den Rapsberry und einer h\"oheren Bildwiederholungsrate gesucht werden. Dabei kam die Idee auf ein 7 Zoll resistiven Touchscreen von Pollin zu verwenden das durch seine Gr\"osse viel Platz f\"ur das Tally Programm bietet und eine sehr gute Touch Bedienung verspricht. Von Vorteil war auch die eigene externe Grafikeinheit die eine hohe Bildwiederholungsrateerm\"oglicht  und ein externer USB- Touchcontroller. 
\par
Jedoch stellte die externe Stromversorgung ein Problem dar, da man jeweils ein Stromkabel f\"ur den Raspberry und eins f\"ur das Display ben\"otigt, damit \"are eine Energiesparende L\"osung ebenfalls nicht m\"oglich.
\par
Deshalb fiel die Entscheidung zugunsten des 3,5 Zoll Touchscreen 4DPi-35 von 4D System aus. Der Touchscreen besitzt eine Aufl\"osung von 480x320 Pixeln und durch die High Speed 48Mhz SPI Verbindung werden hohe und konstante Bildwiederholungsraten erm\"oglicht.
Die Vorteile bei diesem Display waren die kleine Bauform und keine zus\"atzlich ben\"otigte Stromquelle. Ausserdem wird vom Hersteller direkt ein passender Kernel f\"ur Raspbian zur Verf\"ugung gestellt.
\begin{frame}

\begin{lstlisting}
wget http://www.4dsystems.com.au /downloads/4DPi/kernel4dpi_1.3-3_pi2.deb 

sudo dpkg -i kernel4dpi_1.3-3_pi2.deb
\end{lstlisting}

\end{frame}

 Nachdem das erforderliche Paket installiert wurde, musste der Raspberry herunter gefahren werden,  das Display an den GPIO-Ports angebracht werden und der Raspberry wieder hoch gefahren werden. Danach war ein sofortiger Betrieb m\"oglich. Dadurch das das Display direkt als Prim\"ares Display erkannt wird ist kein HDMI Bildschirm mehr notwendig und die Bedienung kann ausschliesslich \"uber den Touchscreen erfolgen.
\par
\subsubsection{Produkte scannen}
Zur vereinfachten Eingabe von Produkten soll ausserdem ein Barcodescanner angebracht werden. Hier sollte entweder ein handesl\"ublicher Barcodescanner per USB an den Raspberry angeschlossen werden, der Barcode mittels USB Webcam und passender Software oder aber ein Barcodescannermodul an den GPIO's des Raspberry als Lösung dienen.
\par
 Da jedoch das vorhandene Barcodescannermodul nur mit 3V betrieben werden kann, des Raspberry aber 5V an den GPIO's zur Verf\"ugung stellt, wurde überlegt eine Webcam an den Raspberry anzuschliessen und mittels einer Software das Bild auf Barcodes zu untersuchen. Die Wahl fiel allerdings auf das Scannen mittels USB Barcodescanner, das dies eine einfacherere Handhabung mit sich bringt und der Barcodescanner direkt als USB Tastatur erkannt wird.
\par
\subsubsection{Wlan für den Raspberry}
Damit am Raspberry nicht immer ein Patchkabel angeschlossen werden muss, wurde der Edimax WLAN USB Stick am Raspberry angeschlossen. Der Edimax WLAN Stick wird automatisch von Raspbian erkannt da schon alle erforderlichen Pakete und Treiber installiert sind, weshalb dieser WLAN Stick empfohlen wird falls man WLAN am Raspberry benutzen m\"ochte. Zun\"achst wurde in der automatische 'Power Saving' Modus abgeschaltet. Hierzu wurde eine Konfigurationsdatei f\"ur den Treiber des WLAN Sticks erstellt.
\begin{frame}

\begin{lstlisting}
sudo nano /etc/modprobe.d/8192cu.conf
\end{lstlisting}

\end{frame}
\par

mit folgendem Inhalt
\begin{frame}

\begin{lstlisting}
options 8192cu rtw_power_mgnt=0 rtw_enusbss=0
\end{lstlisting}

\end{frame}
\par
\par
Um nun eine Verbindung mit dem Wlan herzustellen wurde die folgende Datei
\begin{frame}

\begin{lstlisting}
sudo nano /etc/network/interfaces
\end{lstlisting}

\end{frame}

um folgenden Inhalt ergänzt
\begin{frame}

\begin{lstlisting}
auto lo
iface lo inet loopback
iface eth0 inet dhcp

auto wlan0
allow-hotplug wlan0
iface wlan0 inet dhcp
wpa-ap-scan 1
wpa-scan-ssid 1
wpa-ssid "WLAN-NAME"
wpa-psk "WLAN-PASSWORT"

\end{lstlisting}

\end{frame}

damit eine Verbindung mit dem WLAN Netz hergestellt werden kann.
Anschliessend muss nur noch per
\begin{frame}

\begin{lstlisting}
sudo service networking restart
\end{lstlisting}

\end{frame}
der Netzwerkdienst neu gestartet werden und eine Verbindung mit dem angegebenen WLAN Netz wird hergestellt.

\subsubsection{angepasster Bootscreen}
Während eines Meetings kam die Idee auf den Bootscreen zu verändern und ein Bild an zu zeigen. Dazu wird eine weitere Software benötigt, der 'Frame Buffer Imageviewer'.
\begin{frame}

\begin{lstlisting}
sudo apt-get install fbi
\end{lstlisting}

\end{frame}

Danach muss ein passendes Skript geschrieben werden welches das Bild beim booten anzeigt.
\begin{frame}

\begin{lstlisting}
sudo nano asplashscreen
\end{lstlisting}

\end{frame}

\begin{frame}

\begin{lstlisting}
#! /bin/sh
### BEGIN INIT INFO
# Provides:          asplashscreen
# Required-Start:
# Required-Stop:
# Should-Start:      
# Default-Start:     S
# Default-Stop:
# Short-Description: Show custom splashscreen
# Description:       Show custom splashscreen
### END INIT INFO
 
do_start () {
 
    /usr/bin/fbi -T 1 -noverbose -a /etc/splash.png    
    exit 0
}
 
case "$1" in
  start|"")
    do_start
    ;;
  restart|reload|force-reload)
    echo "Error: argument '$1' not supported" >&2
    exit 3
    ;;
  stop)
    # No-op
    ;;
  status)
    exit 0
    ;;
  *)
    echo "Usage: asplashscreen [start|stop]" >&2
    exit 3
    ;;
esac
 
:
\end{lstlisting}

\end{frame}

Dieses Skript wird nun in den Ordner verschoben damit es beim Start ausgeführt werden kann
\begin{frame}

\begin{lstlisting}
sudo mv asplashscreen /etc/init.d/asplashscreen
\end{lstlisting}
\end{frame}

Als letztes wird das Skript für den Raspberry ausführbar gemacht und als Bootscript registriert
\begin{frame}

\begin{lstlisting}
sudo chmod a+x /etc/init.d/asplashscreen
sudo insserv /etc/init.d/asplashscreen
\end{lstlisting}
\end{frame}

\subsubsection{Backups einrichten}





Wie wir im vorangegangen Text gesehen haben, sind die Zeilenumbche automatisch ganz gut. Neue Abschnitte werden durch eine Leerzeile erzeugt. Wenn man trotzdem eine Zeile in einem Absatz abbrechen will, so geht das natrlich auch. \\  Aber eigentlich sehen diese Umbche seltsam aus und man sollte sie vermeiden. Da ist es doch in der Regel besser, gleich einen neuen Absatz zu beginnen. In Abschnitt \ref{Sonderzeichen} sehen wir, wie eine gute Absatzstruktur aussieht. brigens knnen wir auch einen neuen Seitenbeginn erzwingen.

\newpage

Aber auch das sollte man in der Regel vermeiden. \LaTeX\ bricht Seiten in der Regel selbst vernnftig um, und wenn man spter noch Text einfgt, stehen manuelle Seitenumbrche fast immer an der falschen Stelle.\footnote{Manuelle Seitenumbrche sollten deswegen hchstens dann eingefgt werden, wenn das Dokument vollstndig fertig ist und die Seitenumbrche von \LaTeX\ nicht zufriedenstellend waren.}

\subsection{Sonderzeichen}
\label{Sonderzeichen}

Viele Sonderzeichen haben in \LaTeX\ eine spezielle Bedeutung und drfen nicht einfach so verwendet werden, sondern werden durch spezielle Befehle erzeugt. Das ist ganz hnlich wie HTML. Zum Beispiel ist \& ein Trennzeichen in Tabellen, \$ signalisiert den Beginn und das Ende von mathematischem Text, \% macht eine Zeile zu einem Kommentar, der von \LaTeX\ natrlich ignoriert wird, \{ und \} sind fr die Parametebergabe bei Befehlen reserviert, \_ und \^{} haben ihre Bedeutung bei dem Setzen von mathematischen Formeln und \# ist erforderlich bei selbst definierten Befehlen.


%------------------------------------------------------------------------------------------------

\section{Auflistungen und Aufzhlungen}
\label{Listen}

%------------------------------------------------------------------------------------------------
Wir fassen zusammen, was wir bisher knnen:

\begin{itemize}
 \item Die Schrift anpassen
 \item Den Text ausrichten. Da gab es folgende Mglichkeiten:
    \begin{itemize}
        \item Blocksatz
        \item linksndig
        \item rechtsbndig
        \item zentriert. Zentrierte Text ist vor allem gut um
            \begin{itemize}
                \item Text hervorzuheben
                \item Objekte, wie Tabellen und Grafiken zu zentrieren
            \end{itemize}
    \end{itemize}
 \item Umbrche erzwingen
 \item Sonderzeichen setzen
\end{itemize}

Bisher knnen wir noch nicht:

\begin{enumerate}
 \item Tabellen erzeugen
 \item Grafiken einbinden
 \item mathematischen Text setzen. Dazu gehrt
    \begin{enumerate}
       \item weitere mathematische Sonderzeichen, wie z.B.
           \begin{enumerate}
               \item griechische Buchstaben, etwa $\alpha$, $\zeta$, usw.
               \item spezielle Akzente, wie $\vec{a}$ oder $\ddot{x}$
               \item echte Sonderzeichen, wie $\oplus$ oder $\perp$
               \item alle Mglichen Klammern, wie $\lfloor x \rfloor$
               \item und noch vieles mehr\ldots
           \end{enumerate}
       \item spezielle mathematische Objekte, wie Matrizen
       \item automatische Nummerierung von Formeln
    \end{enumerate}
 \item Referenzen und Bibliographie erzeugen
 \item Prentationsfolien erstellen
\end{enumerate}

Offensichtlich kann man hier ziemlich tief schachteln. Ob das allerdings immer sinnvoll ist?


%------------------------------------------------------------------------------------------------

\section{Tabellen, Grafiken und Gleitobjekte}
\label{TabellenGrafikenFloats}

%------------------------------------------------------------------------------------------------

\subsection{Grafiken}
\label{Grafiken}


\begin{center}
\includegraphics[width=12cm]{TallyLogo.png}
\end{center}

Hier hatten wir Gtig kommt?

\subsection{Tabellen}
\label{Tabellen}



\begin{center}
\begin{tabular}{|l||c|c|c|}
\hline
           & \multicolumn{3}{|c|}{gemessen von Dr. Tex} \\
\cline{2-4}
Name       &  Alter    & Gre   & Gewicht \\
           &  (Jahre)  & (in cm) & (in kg) \\
\hline
\hline
           &   7       &  120    &  25     \\
\cline{2-4}
Andreas    &   10      &  141    &  34     \\
\cline{2-4}
           &   14      &  163    &  50     \\
\hline
           &   6       &  110    &  22     \\
\cline{2-4}
Beate      &   9       &  138    &  32     \\
\cline{2-4}
           &   13      &  156    &  46     \\
\hline
           &   8       &  132    &  30     \\
\cline{2-4}
Tina       &   11      &  151    &  43     \\
\cline{2-4}
           &   15      &  174    &  51     \\
\hline
\end{tabular}
\end{center}

\subsection{Bewegliche Objekte}
\label{Floats}


\begin{figure}[h]
\begin{center}
\includegraphics[angle=-90,width=6cm]{bild.pdf}
\caption{Dasselbe nochmal als Gleitobjekt. \label{bild}}
\end{center}
\end{figure}

%------------------------------------------------------------------------------------------------

\end{document}

%------------------------------------------------------------------------------------------------
